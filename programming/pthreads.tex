\documentclass{beamer}
% \usepackage[T1]{fontenc}
\usepackage{lmodern}
\usepackage{listings}

\title{Pthreads Tutorial}

\usetheme{CambridgeUS}

\usepackage[utf8]{inputenc}

\begin{document}
\textsf{}
\maketitle	
  
  \begin{frame}
  	\frametitle{Synchronization}
  	\begin{itemize}
      \addtolength{\itemsep}{15pt}
      \item Invariants are assumptions made by a program, especially assumptions about
            relationships between sets of variables. E.g., When money is transferred between
            two bank accounts, the sum of the accounts must be the same before and after the
            transfer.
      \item Critical sections are areas of code that affect a shared state.
      \item A critical section can almost always be translated into a data invariant, and vice
            versa. When you remove an element from a queue, for example, you can see the code
            performing the removal as a critical section, or you can see the state of the queue
            as an invariant.
  	\end{itemize}
  \end{frame}

  % Frame 2
  
  \begin{frame}
    \frametitle{Synchronization}
    \begin{itemize}
      \addtolength{\itemsep}{15pt}
      \item Most invariants break momentarily. The trick is to restore the invariant other parts
            of the code encounters them. That is a large part of what "synchronization" is all
            about.
      \item Synchronization protects your program from the fleeting inconsistency. If your code
            locks a mutex whenever it must (temporarily) break an invariant, then other threads
            that rely on the invariant, and which also lock the mutex, will be delayed until the
            mutex is unlocked -- when the invariant has been restored.
      \item "Predicates" are logical expressions that describe the state of invariants needed by
            your code.
    \end{itemize}
  \end{frame}
  
  \begin{frame}
    \frametitle{Intrinsic Loking in Java - The Synchronized Block}
    \begin{itemize}
      \addtolength{\itemsep}{5pt}
      \item Every Java object can implicitly act as a lock for purposes of synchronization
      \item These built-in locks are called \textit{\textbf{intrinsic locks}} or
            \textbf{\emph{monitor locks}}.
      \item The lock is automatically acquired by the executing thread before entering a
            synchronized block and automatically released when control exits the synchronized block,
            whether by the normal control path or by throwing an exception out of the block.
      \item The only way to acquire an intrinsic lock is to enter a synchronized block or method
            guarded by that lock.
      \item Intrinsic locks in Java act as mutexes (or mutual exclusion locks), which means that
            at most one thread may own the lock
    \end{itemize}
  \end{frame}
    
  \begin{frame}[fragile]
    \frametitle{Reentrancy}
    \begin{itemize}
      \item Reentrancy means that locks are acquired on a per-thread rather than per-invocation basis
      \item If a thread tries to acquire a lock that it already holds, the request succeeds
      \item Java's intrinsic locks are reentrant
      \item By default, pthreads (POSIX threads) mutexes are granted on a per-invocation basis, and
            hence are not reentrant. pthreads provide a reentrant variety as follows:
    \end{itemize}
    \begin{verbatim}
      pthread_mutex_t mutex;
      pthread_mutexattr_t mutex_attr;

      pthread_mutexattr_init(&mutex_attr);
      pthread_mutexattr_settype(&mutex_attr,
                                PTHREAD_MUTEX_RECURSIVE);
      pthread_mutex_init(&mutex, &mutex_attr);
    \end{verbatim}
  \end{frame}
\end{document}