\documentclass[12]{beamer}

\usetheme{CambridgeUS}
\usepackage{tcolorbox}

%
% personal definitions of commands
%
\newcommand{\LBD}{\boldsymbol{\lambda}}

\begin{document}

\begin{frame}
  \frametitle{Exponential Distribution (Continuous Process)}
  \begin{block}{Rate}
  A continuous process (as opposed to discrete process) whose probability is uniform in time can be completely described by just one number -- the \textbf{average rate of success} which we'll call $\boldsymbol{\lambda}$.
  \end{block}
  \vspace{7.5pt}
  The rate $\LBD$ can be determined by counting the number of successful events during a long time interval, and then dividing it by the duration. For example, if 300 events happen during 100 minutes, then the rate $\LBD$ is 3 events per minute. The observed number will be different in another trial. But in the limit of a very long time interval, one will find essentially the same rate, independent of which specific long interval one uses.
  \center\tcbox{Expected number of events in time \textit{t} = \textit{$\LBD$t}}
\end{frame}

\end{document}