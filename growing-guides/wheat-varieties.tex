\documentclass[12]{article}

\title{Classification of Indian Wheats}
\author{http://agropedia.iitk.ac.in/content/classification-indian-wheats}
\date{}

\usepackage{xcolor}
\renewcommand{\familydefault}{\sfdefault}

\addtolength{\oddsidemargin}{-.875in}
\addtolength{\evensidemargin}{-.875in}
\addtolength{\textwidth}{1.75in}
\addtolength{\topmargin}{-.875in}
\addtolength{\textheight}{1.75in}

\begin{document}

\maketitle

Wheat is an annual plant of Grammeae family. It belongs to genus Triticum. Although as many as 18 species of wheat have been described and recognized, only a few are of agricultural importance. They are discussed as under.

\begin{enumerate}

\item \textbf{\color{brown}Emmer wheat (Triticum dicoccum Schuh L.)}: This is reported to be grown in certain areas of southern India i.e. Maharashtra, Tamilnadu and Karnataka and Andhra Pradesh. It is good for the South Indian dish upma.

\item \textbf{\color{brown}Macaroni wheat (Triticum durum des/)}: Cultivation of macaroni wheat in India is considered to be very old. It is the best wheat for drought conditions or under restricted irrigation conditions of Punjab, M.P., Karnataka, Tamil Nadu, Gujarat, West Bengal and Himachal Pradesh. It is used for semolina (i,e. suji or rawa, semya and sphagetti).

\item \textbf{\color{brown}Common bread wheat (T. vulgare Hist)}: It is a typical wheat of alluvial soils of Indo-Gangetic plains i.e. Punjab, U.P., Bihar and parts of Rajasthan. The bulk of Indian crop, therefore, consists of this type. The prominent varieties are K-65, K.68, C-13, Pb-591 and C-306.

\item \textbf{\color{brown}Indian dwarf wheat (Triticum spherococcum Mihi)}: This belongs to the club wheat of Western countries. This is grown in limited areas of India and Pakistan. These are characterised by very short and compact heads having shorter grains. They contain 42 chromosomes.

\item \textbf{\color{brown}Mexican dwarf wheat (Triticum aestivum)}: This is presently grown in almost all the wheat growing zones. This wheat was introduced in India by Dr N.E. Borlaug of Mexico. Now, it is the most widely grown wheat species in India. It covers more than 87\% of the total wheat growing area of India, followed by durum wheat  and dicoccum wheat. The common wheat (T. aestivum) which is good for chapati making and bakery products is grown in whole of North India and also in central and South India.

\end{enumerate}

\section{Indian Varieties}

The origin of these wheat varieties date back to several thousand years, and are not only high on nutrition value... but have high fibre and low gluten. Each come with their unique characteristic of colour, flavour, taste, texture and medicinal value besides yield and pest \& disease resistance.
They are as nature intended them to be in their pure form, without any 'engineering' done by man. They are all low gluten with the gluten structure also being different than the gluten found in modern-day highly engineered and hybridised wheat we get these days. And hence, these varieties are not likely to cause any of the symptoms of gluten intolerance that modern-day wheat causes.

They were all grown and harvested as a Rabi crop and our granaries are now stocked with them. Here's the list of native heirloom wheat diversity that we are happy to offer you now:

\begin{enumerate}

\item KATHIYA (a red Durum wheat): Our oldest variety which many of you have been consuming for a long time now ! That beautiful slightly pink coloured wheat with a lovely aroma.

\item KHAPLI (a.k.a. EMMER): One of the earliest crop to be domesticated by man in the Fertile Crescent in Central Asia, Emmer/ Khapli (Triticum diococcum) is a husked wheat with 2 grains in each husk. A very low gluten, high fibre wheat.. this one is known for its drought and pest-disease resilience.

\item PYGAMBARI: Popularly known as the sugar-free wheat, it is a cute round/ spherical shaped grain (Triticum spherococcum). This one's origin has been traced back to the Indus Valley Civilisation (almost 4500 years ago), and hence believed to be India's first wheat variety. An extremely good choice for diabetics because of its high fibre and low GI. One of the preferred variety to make wheat mylk out of (esp. for vegans).

\item KALIBAL: Named so because of it gorgeous looking black hair panicles, this one has dark red grains.

\item BANSI (the quintessential Durum): This one is a high grade amber-coloured Durrum wheat (Triticum Durum) known for its plumpness, appearance, consistency and hardness. The very best variety for making semolina and pasta.

\item PISI: This is a bold grained, soft white wheat and is one of the parents of the Sharbati wheat selection that was developed in the 1960s. We've been bowled over by this flour!

\end{enumerate}


\end{document}