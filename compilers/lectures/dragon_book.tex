\documentclass[12]{beamer}

\usetheme{Madrid}
\usecolortheme{rose}

\usepackage[utf8]{inputenc}
\usepackage[T1]{fontenc}
\usepackage[ngerman]{babel}
\usepackage{parskip}

\title{Compilers, Principles, Techniques and Tools}
\author{Prasad}
\date

\begin{document}

\maketitle

\setbeamertemplate{footline}[page number]{}

\begin{frame}
\frametitle{Syntax-directed Definition}

\end{frame}

\begin{frame}
\frametitle{Syntax-Directed Translation}

Syntax-directed translation is a notation to attach rules or program fragments to productions in a grammar. 

\end{frame}

%
% next frame
%
\begin{frame}
\frametitle{Syntax-Directed Translation ...contd}

\begin{block}{synthesized attributes}
An attribute is said to be synthesized if its value at a parse-tree node 'N' is determined from attribute values at its children and the node 'N' itself. Can be evaluated during a single bottom-up traversal of a parse tree.
\end{block}

A parse tree showing the attribute values at each node is called an annotated parse tree
\end{frame}

%
% next frame
%
\begin{frame}
\frametitle{Syntax-Directed Translation ...contd}
\begin{block}{semantic actions}
Program fragments embedded within (grammar) production bodies
\end{block}

The implementation of a translation scheme must ensure that semantic actions are performed in the order they would appear during a postorder traversal of a parse tree.

(When drawing a parse tree for a translation scheme, we indicate an action by constructing an extra child for it, connected by a dashed line to the node that corresponds to the head of the production).

\end{frame}

%
% Frame on Parsing
%
\begin{frame}
\frametitle{Parsing}
Parsing is the derivation of a syntactic structure for a program (a sentence in the language), fitting the non-terminals (lexemes) into the grammatical model (CFG) of the programming language. In other words, for a stream of words 's' and a grammar G, the parser tries to build a constructive proof that 's' can be derived in 'G'.
\\~\\
If the parser determines that the input stream is a valid program, it builds a concrete model (intermediate representation, IR) of the program for use by later phases of compilation. A parse tree is one such IR, and hence is also knows as 'concrete syntax tree'. Parser implementations can be classified as:
\begin{itemize}
\item table-driven
\item direct-coded
\item hand-coded
\end{itemize}
\end{frame}

%
% Parsing Methods
%
\begin{frame}
\frametitle{Parsing Methods}
Most parsing methods fall into one of two classes.
\begin{itemize}
\item top-down
\item bottom-up
\end{itemize}
These terms refer to the order in which the nodes of a parse tree are constructed.  
\begin{block}{top-down parsers}
In top-down parsers construction starts at the root and proceeds towards the leaves. In other words, top-down parsers try to match the input stream against the productions of the grammar by predicting the next production (rule) at each point.
\end{block}
\begin{block}{bottom-up parsers}
In bottom-up parsers, construction starts at the leaves and proceeds towards the root. They work from low-level detail (sequence of words), and accumulate context until the derivation (production) is apparent.
\end{block}
\end{frame}


%
% parsing methods continued
%
\begin{frame}
\frametitle{parsing methods ...contd}
Efficient parsers can be constructed more easily by hand using top-down methods. Bottom-up parsing, however can handle a larger class of grammars and translation schemes, so software tools for generating parsers directly from grammars often use bottom-up methods.
\end{frame}

\end{document}