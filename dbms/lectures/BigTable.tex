\documentclass[12pt]{beamer}
\title{BigTable: A Distributed Storage System for Structured Data}

\usetheme{CambridgeUS}

\begin{document}
\maketitle

\begin{frame}
  \frametitle{Rows}
  \begin{itemize}
    \addtolength{\itemsep}{10pt}
    \item Row keys in a table are arbitrary strings (upto 64KB in size)
    \item Every read or write of data under a single row  key is atomic (regardless of the number of columns
          being read or written to)
    \item BigTable maintains data in lexicographic order by row key
    \item The row range for a table is dynamically partitioned
    \item Each row range is called a {\color{red}{tablet}}, which is the unit of distribution and load balancing
  \end{itemize}
\end{frame}

\begin{frame}
  \frametitle{Column Families}
  \begin{itemize}
    \addtolength{\itemsep}{5pt}
    \item Column keys grouped into sets are called {\color{red}{column families}}
    \item Column families are the basic unit of access control
    \item All data stored in a column family is usually of the same type (facilitates compression of data in
    	  the same column family)
	\item Fewer column families (in hundreds), but unbounded number of columns in each family
	\item A column key is named using the syntax {\color{red}{family:qualifier}}. Family names must be printable,
		  but qualifiers may be arbitrary strings
    \item Access control and both disk and memory accounting are performed at the column-family level
  \end{itemize}
\end{frame}

\end{document}