\documentclass[12]{beamer}

\usetheme{Singapore}

\usepackage[utf8]{inputenc}
\usepackage[T1]{fontenc}
\usepackage[ngerman]{babel}

\title{Database Concepts}
\author{Prasad}
\date

\begin{document}

\maketitle

\begin{frame}
\frametitle{What is a Transaction}

A basic unit of consistent and reliable computing (\"Ozsu and Valduriez) \\
\vspace{5mm}
Proved to be the major paradigm for synchronization and recovery (Haerder and Reuter)

\end{frame}

\begin{frame}
\frametitle{Consistency}
The notion of consistency can be categorized as:
\section{•}
\end{frame}

\begin{frame}
\frametitle{Isolation and Consistency}

The concepts 'Isolation' comes into picture when multiple transactions (from the same user or different users) have to be executed in parallel. \\
\vspace{5mm}
The techniques that achieve isolation are known as \textit{synchronization} or \textit{Concurrent Control Techniques}. In the absence of consistency guarantees, values in database will be rendered inconsistent, i.e., plain wrong and dangerous.
\end{frame}

\begin{frame}
\
\end{frame}

\end{document}