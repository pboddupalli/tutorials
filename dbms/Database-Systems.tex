\documentclass[12]{beamer}

\usetheme{CambridgeUS}

\begin{document}

\section{The Relational Model of Data}

\subsection{An Overview of Data Models}

% slide 1
\begin{frame}
\frametitle{Relational Model of Data}
\begin{block}{What is a Data Model ?}
\vspace{10pt}
A data model is a notation for describing data or information. It generally consists of:
\begin{itemize}
\item \textit{\textbf{Structure of the Data}} refers to higher level structures used to represent data. E.g., Tables, XML, JSON
\item \textbf{\textit{Operations on the data}} Operations that manipulate or fetch data from structures of the data model. E.g., Union, Select, Projection
\item \textbf{\textit{Constraints on the Data}} a way to describe limitations on what the data can be.
\end{itemize}
\end{block}
\end{frame}

\end{document}

